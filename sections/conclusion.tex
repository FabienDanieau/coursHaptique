\section{Conclusion} 
\begin{frame}{Conclusion}
	\begin{itemize}
		\item Haptic: science du toucher
		\begin{itemize}
		\item Perception tactile
		\item Perception kinésthésique
		\end{itemize}
		\item Sens du mouvement: perception de son corps dans l'espace
		\begin{itemize}
		\item Perception haptique
		\item Système vestibulaire
		\end{itemize}
		\item Interfaces haptiques / Simulateur de mouvement
		\begin{itemize}
		\item Appareils mécaniques qui stimulent ces systèmes perceptifs
\end{itemize}
		\item Algorithmes de rendu
		\begin{itemize}
		\item boucles de contrôle des interfaces haptiques
		\end{itemize}
		\item Multiples applications: médical, enseignement, jeu, réalité virtuelle, etc.
	\end{itemize}
\end{frame}	 


\begin{frame}{Conférences / Journaux / Ressources}
	\begin{itemize}
	\item Haptique
	\begin{itemize}
	\item Haptics Symposium
	\item Eurohaptics
	\item World Haptics
	\item IEEE Transactions on Haptics
	\end{itemize}
	\item IHM
	\begin{itemize}
	\item \href{https://www.youtube.com/user/acmsigchi}{UIST}
	\item CHI
	\item SIGGRAPH
	\end{itemize}
	\item Web
	\begin{itemize}
	\item https://twitter.com/HapticsClub
	\item https://hapticsif.org
	\item https://haptipedia.org/
	\end{itemize}
	\end{itemize}

\end{frame}

\begin{frame}[standout]
  Merci! \\
  Question? \\
  fabien.danieau@interdigital.com
\end{frame} 